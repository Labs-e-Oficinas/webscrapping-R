\documentclass[11pt]{beamer}
\usepackage{beamerthemesplit}
\usepackage[brazilian]{babel}
\usepackage[utf8]{inputenc}
\usepackage{amssymb}
\usepackage{graphicx}
\let\oldemptyset\emptyset
\let\emptyset\varnothing

\title{Captura de dados da internet, sistematização o e análise de ”Big data”}

\subtitle{Aula 2}

\author[Leonardo Sangali Barone]{Leonardo Sangali Barone\\MQ 2015 - UFMG}
\date[04 de Agosto de 2015]{04 de Agosto de 2015}

\begin{document}
\frame{\titlepage}

\section{Agenda}

\begin{frame}
	\frametitle{Agenda}
	\begin{itemize}
		\item Listas e tipos de objetos no R (lousa)
		\item Brevíssima introdução a HTML (pdf)
		\item Atividade 2
		\item Atividade 3
	\end{itemize}
\end{frame}

\section{Introdução a HTML}

\begin{frame}
	\frametitle{Introdução a HTML}
	HTML significa HyperText Markup Language. Hypertext (hipertexto) significa "texto que contém links."
\newline\\
	Uma markup language (linguagem de marcação) é uma linguagem de programação usada para fazer com que o texto faça mais do que simplesmente ficar parado em uma página: ela pode transformar textos em imagens, links, tabelas, listas, e muito mais. HTML é a linguagem que vamos aprender.	
\newline\\
(Texto retirado de www.codeacademy.com)
\end{frame}

\begin{frame}
	\frametitle{Introdução a HTML}
	A grande maioria das páginas das quais queremos obter os dados estão escritas em HTML.
	\newline\\
	Nossa atividade será criar páginas extremamente simples em HTML para compreender o básico de um documento HTML.
	\newline\\
	Precisamos apenas um Notepad e um navegador para começar a aprender.
\end{frame}

\begin{frame}
	\frametitle{Introdução a HTML}
	As marcações em um documento HTML são chamadas de \emph{tags}.
	\newline\\
	Cada tag começa com o símbolo $<$ e termina com $>$.
	\newline\\
	Exemplo: $<$title$>$
\end{frame}

\begin{frame}
	\frametitle{Introdução a HTML}
	Em geral as tags vem em pares: uma de início e outra de encerramento.
	\newline\\
	Início: $<$title$>$\\
	Encerramento: $<$/title$>$
\end{frame}

\begin{frame}
	\frametitle{Introdução a HTML}
	As tags podem conter atributos, que, por sua vez, recebem valores
	\newline\\
	Exemplo: $<$tag atributo = ``valor''$>$ $<$/title$>$
	\newline\\
	Os atributos definem as características das tags (por exemplo, estáticas) e também servem para a programadora classificar a tag dentre as várias tags do mesmo tipo.
\end{frame}

\begin{frame}
	\frametitle{Introdução a HTML}
	Links, por exemplo, ficam contidos na tag $<$a$>$ $<$/a$>$ e recebem o atributo href, que contém o endereço para o qual apontam:
	\newline\\
	Exemplo: $<$a href$=$"http://pretocafe.com.br/"$>$ $<$/a$>$
\end{frame}


\end{document}

