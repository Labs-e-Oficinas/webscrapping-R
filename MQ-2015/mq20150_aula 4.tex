\documentclass[11pt]{beamer}
\usepackage{beamerthemesplit}
\usepackage[brazilian]{babel}
\usepackage[utf8]{inputenc}
\usepackage{amssymb}
\usepackage{graphicx}
\let\oldemptyset\emptyset
\let\emptyset\varnothing

\title{Captura de dados da internet, sistematização o e análise de ”Big data”}

\subtitle{Aula 4}

\author[Leonardo Sangali Barone]{Leonardo Sangali Barone\\MQ 2015 - UFMG}
\date[06 de Agosto de 2015]{06 de Agosto de 2015}

\begin{document}
\frame{\titlepage}

\section{Limitações do R como Web Client}

\begin{frame}
	\frametitle{Limitações do R como Web Client}
	R fornece funcionalidades básicas bastante adequadas para obtenção de 		dados via HTTP.
	\newline\\
	Entretanto, as funções básicas não permitem controlar aspectos essenciais 		do uso de HTTP.
	\newline\\
	Vejamos algumas das limitações.
\end{frame}

\begin{frame}
	\frametitle{Limitações do R como Web Client}
	\begin{itemize}
		\item há um número grande outros protocolos de comunicação via web 			(HTTPS, FTP, FTPS, etc) além de HTTP.
		\item não contempla funcionalidades básicas como postar em 
		formulários.
		\item não é possível fazer upload de arquivos (por exemplo, em FTP).
		\item não permite obtenção de dados disponibilizados via HTTPS, que é 			uma variação mais segura do protocolo HTTP (encriptação) bastante 			utilizada (ex: Google).
	\end{itemize}
\end{frame}

\section{RCurl}

\begin{frame}
	\frametitle{RCurl}
	RCurl é um pacote desenvolvido para superar tais limitações.
	\newline\\
	RCurl utiliza uma biblioteca em linguagem C (libcurl) para executar
	funções de Web Client.
	\newline\\
	Curl = ``Client for URL Resources''
	\newline\\
	Funções principais: getURL(), getForm() e postForm().
\end{frame}

\begin{frame}
	\frametitle{RCurl - funções}
	getURL(): semelhante às funções download.url() ou readLines(), com a 		qual havíamos trabalho até então, serve para a obtenção de conteúdo 	
	(normalmente o corpo da página HTML) via HTTP.
	\newline\\
	Exemplo:page = getURL(``http://www.fafich.ufmg.br/'')
\end{frame}

\begin{frame}
	\frametitle{RCurl - funções}
	Adicionalmente, getURL também recebe como argumento um vetor de URLs e 		retorna uma lista com seus conteúdos.
	\newline\\
	Exemplo: page = getURL(c(url1, url2, url3))
\end{frame}

\begin{frame}
	\frametitle{RCurl - funções}
	A função getURL() contém diversas opções para se adequar ao tipo de
	protocolo requisitado e às características do Web server.
	\newline\\
	Exemplo: para páginas em que há ``redirecionamento'', podemos usar a opção
	followlocation = TRUE
	\newline\\
	page = getURL(url, followlocation = TRUE)
\end{frame}

\begin{frame}
	\frametitle{RCurl - funções}
	Há mais uma centena de opções atualmente para cutomizar os resquests  que
	a função getURL a um Web server.
	\newline\\
	names(getCurlOptionsConstants())
\end{frame}

\begin{frame}
	\frametitle{RCurl - funções}
	E se a URL não existir?
	\newline\\
	A função url.exists() retorna TRUE ou FALSE e indica se a URL existe e, 	portanto, podemos capturá-la. Esta função é particularmente útil quando 	usamos um loop para obter o conteúdo de vários URLs ao mesmo tempo.
	followlocation = TRUE
	\newline\\
	Exemplo:\\
	url.exists(``http://www.urlfalsa.gov.ar'')\\
	$[1]$ FALSE
\end{frame}

\begin{frame}
	\frametitle{RCurl - funções}
	getForm() e postForm(): Ambas funções servem para submetermos documentos 
	de HTML via HTTP com a finalidade de capturar seu resultado. São usadas 
	sobretudo com formulários.
\end{frame}

\begin{frame}
	\frametitle{RCurl - funções}
	Formulários: menus, checkboxes, caixas de rolagem e etc. Permite ao 
	usuários definir valores para variáveis .
	\newline\\
	Exemplos de formulários: buscador Google, consulta de CEP nos correios 
	ou de municípios no IBGE.
\end{frame}

\end{document}


