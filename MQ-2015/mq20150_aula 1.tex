\documentclass[11pt]{beamer}
\usepackage{beamerthemesplit}
\usepackage[brazilian]{babel}
\usepackage[utf8]{inputenc}
\usepackage{amssymb}
\usepackage{graphicx}
\let\oldemptyset\emptyset
\let\emptyset\varnothing

\title{Captura de dados da internet, sistematização o e análise de ”Big data”}

\subtitle{Aula 1}

\author[Leonardo Sangali Barone]{Leonardo Sangali Barone\\MQ 2015 - UFMG}
\date[03 de Agosto de 2015]{03 de Agosto de 2015}

\begin{document}
\frame{\titlepage}

\section{Apresentação}
\begin{frame}
	\LARGE{Apresentação}
\end{frame}

\begin{frame}
	\frametitle{O que vamos fazer em cinco dias?}
	A resposta é bastante simples: aprender a programar. E programar com o objetivo de coletar dados sem clicar e arrastar inúmeras vezes (centenas? milhares? milhões?).
	\newline\\
	Para capturar dados na internet e sistematizar grandes quantidades de dados nossas únicas alternativas são:
	\begin{itemize}
		\item Aprender uma linguagem de programação;
		\item Usar um software proprietário desenhado para este propósito específico.
	\end{itemize}
\end{frame}

\begin{frame}
	\frametitle{O que vamos fazer em cinco dias?}
	O grande problema de usar software proprietário é que estamos limitados ao que ele foi desenhado para fazer (além do fato de que software livre é bem mais legal!).
	\newline\\
	Nossa alternativa será, portanto, aprender a usar R para a captura de dados e construção de conjuntos de dados que estejam prontos para aplicarmos os nossos conhecimentos de análise.
	\newline\\
	Além de R, vamos trabalhar também com algumas outras linguagens/estruturas de dados, como HTML e XML.
\end{frame}

\begin{frame}
	\frametitle{O que vamos fazer em cinco dias?}
	Há vários ganhos em aprender a programar em R:
	\begin{itemize}
		\item É software livre e acessível a todo mundo
		\item Há uma comunidade muito e grande e crescendo incrivelmente rápido
		\item É uma linguagem de alto nível voltada a objetos e muito do que você vai aprender aqui é próximo ou quase idêntico a qualquer linguagem de programação (você vai poder dizer que sabe algo de programação!)
	\end{itemize}
\end{frame}

\begin{frame}
	\frametitle{O que vamos fazer em cinco dias?}
	Há vários ganhos em aprender a programar em R:
	\begin{itemize}
		\item É uma linguagem que foi desenvolvida para análise de dados. Uma vez que a captura de dados termina, já podemos começar a análise, vizualização e inferência dos dados no próprio R.
		\item É uma das linguagens mais populares nas ciências sociais, crescentemente popular no Brasil, e bastante mais poderosa do que os softwares que estamos acostumados a usar (SPSS, SAS e Stata, por exemplo) nas ciências sociais.
		\item Já falei que é um software livre e acessível a todo mundo?		
	\end{itemize}
\end{frame}

\begin{frame}
	\frametitle{O que o curso não é o que não vamos fazer?}
	Este NÃO é um curso de métodos de pesquisa. Também NÃO é um curso de análise de dados. É um curso de técnica de coleta de dados usando programação em R.
	\newline\\
	Nada do que vamos aprender aqui substitui um bom desenho de pesquisa, adequado ao problema de investigação.
	\newline\\
	Tampouco é um curso de métodos quantitativos. A coleta massiva de dados pode ser usada com várias finalidades e não tem necessariamente relação com a aplicação de análise matemática ou estatística.
\end{frame}

\begin{frame}
	\frametitle{O que o curso não é o que não vamos fazer?}
	Não vamos aprender outros usos do R - por exemplo, testes de hipóteses, modelos lineares, análise multivariada, etc. Vamos aprender um bocado de R, mas vamos nos limitar ao que é necessário ao curso.
	\newline\\
	A boa notícia é que o que é necessário ao curso é o suficiente para você se virar depois autonomamente ou para engatar um curso intermediário de R. Tenho várias recomendações para quem precisar.
\end{frame}

\begin{frame}
	\frametitle{Ajustando (um pouco) as expectativas}
	Temos apenas 15 horas. E vamos fazer um conjunto limitados de atividade. Eles representam soluções bastante comum para vários dos problemas que vocês vão enfrentar nas suas pesquisas.
	\newline\\
	Assim, tão importante quanto terminar as atividades, é compreender que tipo de soluções que elas representam e incorporá-las ao seu repertório de soluções.
	\newline\\
	A captura de dados da internet tem algo de ``artesanal'' e cada situação é um pouco única. O mais importante é saber que há classes de soluções, adaptáveis a cada situação.  
\end{frame}

\begin{frame}
	\frametitle{Ajustando (um pouco) as expectativas}
	IMPORTANTE: há situações em que pode não ser possível, ou que será muito difícil, capturar os dados.
	\newline\\
	Há dados que têm acesso restrito por Lei ou têm interesse comercial e há programadores e desenvolvedores trabalhando justamente para não permitir a captura massica de dados.
	\newline\\
	Exemplos:
	\begin{itemize}
		\item Sites com captcha, tais como Receita Federal (dados de empresas, Correios (CEPs e logradouros), etc.  
		\item Dados do facebook (há um pacote para dados do facebook, mas o acesso de qualquer usuário é limitado)
		\item Sites que impõem limites à captura -- Google Maps e Twitter, que permitem um uso gratuito finito de seus serviços  
	\end{itemize}
\end{frame}

\section{Objetivos e Tópicos do curso}
\begin{frame}
	\LARGE{Objetivos e Tópicos do curso}
\end{frame}

\begin{frame}
	\frametitle{Objetivos}
	\begin{itemize}
		\item Introduzir o aluno à linguagem R;
		\item Fornecer um panorama sobre a construção de sites e de linguagens  de programação utilizadas na internet
		\item Capacitar o aluno a coletar dados da internet de forma automatizada, via programação de “robôs”.
	\end{itemize}
\end{frame}

\begin{frame}
	\frametitle{Tópicos do curso por dia}
	\begin{enumerate}
		\item Introdução e captura de tabelas (Portal da Transparência)
		\item Download de arquivos em massa (TSE) e HTML (Construindo seu site!)
		\item XML e captura de notícias (Portal de Notícias - Folha? Globo?)
		\item Preenchimento de formulários (Google!) e pacote RCurl (tópico tenso do curso!)
		\item Propostas: (Introdução a Machine Learning) ou (Grandes bases de dados no R)
	\end{enumerate}
\end{frame}

\section{Materiais, atividades e avaliação}
\begin{frame}
	\LARGE{Materiais, atividades e avaliação}
\end{frame}

\begin{frame}
	\frametitle{Materiais e atividades}
	A ideia do curso é que seja extremamente aplicado e que vocês quebrem bastante a cabeça.
	\newline\\
	Meu objetivo é falar menos do que falei no ano passado e que vocês trabalhem mais.
	\newline\\
	Todos os dias faremos duas atividades, uma tutorada e a outra será por conta própria. Elaborei novos tutoriais para as atividades e vocês devem recorrer a ele durante e após o curso. Eles compõem uma tímida ``apostila'' do curso.
\end{frame}

\begin{frame}
	\frametitle{Materiais e atividades}
	 Assim, dividiremos nosso tempo em quatro:
	\begin{enumerate} 
		\item Uma breve exposição do problema e das linguagens que vamos usar durante o dia.
		\item Uma aplicação em sala, sem o acompanhamento de vocês
		\item A repetição do exemplo do tutorial (atividades), em duplas ou trios 
		\item A elaboração de uma atividade de captura de dados semelhante ao exemplo do tutorial, em duplas ou trios 
	\end{enumerate}
\end{frame}

\begin{frame}
	\frametitle{Avaliação}
	A avaliação é composta por duas partes:
	\begin{enumerate} 
		\item Participação em sala (40\%)
		\item Atividade de captura de dados realizada ao final do dia (60\%)
	\end{enumerate}
	Basicamente, se vocês vierem 4 ou 5 dias e participarem ativamente do curso partirão de 40\% da nota.Normalmente eu não ligo para presença em cursos, mas, por se tratar de um curso de curta duração e exclusivamente aplicado, a presença em sala de aula é insubstituível. 
\end{frame}

\begin{frame}
	\frametitle{Avaliação}
	As atividades que realizarmos ao final das aulas deverão ser entregues e avaliadas. A entrega da atividade completa, ainda que com ínumeros erros e sem sucesso, por si só garante metade da nota.
	\newline\\
	Quero que vocês não se preocupem com entregas de scripts que funcionam, mas em de fato fazerem o melhor que puderem. Assim, quem participar das aulas e entregar tudo já terá 70\% da nota, independentemente de ter obtido êxito na elaboração das atividades ou não.
	\newline\\
	Atividades bem elaboradas ganham ``nota cheia''. As datas das entregas das atividades combinaremos ao longo do curso e de acordo com a dificuldade de vocês, tendo como limite os prazos do MQ 2015.
\end{frame}

\begin{frame}
	\LARGE{Dúvidas, angústias, críticas, sugestões ou reclamações?}
\end{frame}

\end{document}

